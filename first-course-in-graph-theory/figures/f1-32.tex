\begin{figure}[h]
	\centering
	%P3
	\begin{subfigure}[b]{.3\textwidth}
		\[P_{3}:
		\raisebox{-.5\height}
		{
			\begin{tikzpicture}
				\foreach \i in {0,1,2} {
					\vertex (\i) at (0,\i){};
				}
				\path
					(0) edge (1)
					(1) edge (2)
				;
			\end{tikzpicture}
		}\]
	\end{subfigure}%
	%K3
	\begin{subfigure}[b]{.3\textwidth}
		\[K_{3}:
		\raisebox{-.5\height}
		{
			\begin{tikzpicture}
				\vertex (0) at (0,1.5){};
				\vertex (1) at (.5,.5){};
				\vertex (2) at (1,1.5){};
				\path
					(0) edge (1)
					(0) edge (2)
					(1) edge (2)
				;
			\end{tikzpicture}
		}\]
	\end{subfigure}%
	%P3xK3
	\begin{subfigure}[b]{.3\textwidth}
		\[P_{3} \times K_{3}:
		\raisebox{-.5\height}
		{
			\begin{tikzpicture}
				\foreach \i in {0,1,2} {
					\vertex (u\i) at (0,\i){};
					\vertex (v\i) at (.5,\i-.5){};
					\vertex (w\i) at (1,\i){};
				}
				\path
					(u0) edge (u1)
					(u0) edge (v0)
					(u0) edge (w0)
					(u1) edge (u2)
					(u1) edge (v1)
					(u1) edge (w1)
					(u2) edge (v2)
					(u2) edge (w2)
					(v0) edge (v1)
					(v0) edge (w0)
					(v1) edge (v2)
					(v1) edge (w1)
					(v2) edge (w2)
					(w0) edge (w1)
					(w1) edge (w2)
				;
			\end{tikzpicture}
		}\]
	\end{subfigure}
	\caption{The Cartesian product of two graphs}
\end{figure}